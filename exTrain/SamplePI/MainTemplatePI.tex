
%*******************************************************************************************
% File ini merupakan file template LaTEX untuk Penulisan Ilmiah pada Universitas Gunadarma
%% Programmer  : Aldino Septa Nugraha
%% E-mail  : aldi_tob_2000@yahoo.com
%% Pembuatan program template makalah ini dibuat dengan menggunakan :
%% Ms. Windows XP SP1 
%% TeXnicCenter
%% MikTeX 2.5.4
%*******************************************************************************************
%berikut langkah-langkah untuk mendapatkan hasil akhir PI
%=======================================================================================================
% Untuk platform Windows, software berikut harus sudah diinstall :
% MiKTeX (Latex versi Windows). Dapat didownload dari situs http://miktex.org/
% Untuk kemudahan disarankan untuk menginstall editor khusus Latex untuk versi Windows yaitu:
% WinEdt (dapat didownload dari http://www.winedt.com
% Untuk compiling, pada directory file latex berada di command promt ketikkan :
%  latex MainTemplatePI
%  bibtex MainTemplatePI
%  makeindex subject
%  latex MainTemplatePI
%  latex MainTemplatePI
%
%  Lalu untuk mendapatkan hasil output dalam bentuk pdf kompile dengan cara mengetikkan
%  pdflatex MainTemplatePI
%
%  Nanti akan terbentuk file baru bernama MainTemplatePI.pdf
%
%  Bagi yang memakai WinEdt untuk compiling cukup dengan cara :
%  tekan tombol Latex
%  tekan tombol bibtex
%  tekan tombol makeindex
%  tekan tombol Latex
%  tekan tombol Latex
%  tekan tombol Pdf Latex/Pdf Texify
%=======================================================================================================
%  Jangan lupa bila ada suatu perubahan pada file *.tex ataupun *.bib harus dikompile dari langkah
%  pertama dan jangan langsung ke langkah pdflatex %
%=======================================================================================================

\documentclass[a4paper,12pt,oneside,makeidx]{SamplePI}
\usepackage[bahasa]{babel}
\usepackage[T1]{fontenc}
\usepackage[latin1]{inputenc}
\setcounter{secnumdepth}{3}
\setcounter{tocdepth}{3}
\usepackage{array}
\usepackage{longtable}
\usepackage{graphicx}
\usepackage{setspace}
\usepackage{multind}

%\makeindex{subject}

\centerchapter
\makeatletter
\doublespacing
\makeatother
\parindent 3.0em
%===================================================================
\setlength{\textwidth}{15.0cm}
\setlength{\evensidemargin}{2.5cm} % outer/right margin
\setlength{\topmargin}{0.3cm}      % top margin
\setlength{\footskip}{2.5cm}         % distance between text and foot
\setlength{\textheight}{\paperheight}
\addtolength{\textheight}{-\topmargin}
\addtolength{\textheight}{-\headheight}
\addtolength{\textheight}{-\headsep}
\addtolength{\textheight}{-\footskip}
\addtolength{\textheight}{-4cm}   % bottom margin
%==============================================================================
%% Bagian Pemenggalan Kata Yang Tidak Sempurna
%%\hyphenation{Universitas} \hyphenation{Gunadarma}
%==============================================================================
\begin{document}
\nocite{*}
%#start bagian administrasi #==================================================
% bagian muka sebelum isi, umumnya halaman administrasi, kalau tidak digunakan
% dapat diberikan tanda % didepannya
%cover
\pagestyle{empty} \pagenumbering{roman} \setcounter{page}{1}
\documentclass{article}
\usepackage{graphicx}

\begin{document}
\newpage
\addcontentsline{toc}{chapter}{HALAMAN JUDUL}
\begin{center}
\bfseries
 { \Large UNIVERSITAS GUNADARMA}
\vspace{0.75cm}

\begin{figure}[h]
\begin{center}
\includegraphics[height=5cm,width=5cm]{LogoGunadarma.jpg}
\end{center}
\end{figure}

\vspace{1.0cm}

 {\large JUDUL TULISAN}

\bfseries \vspace{0.5cm}
\begin{tabular}{ll}
Nama& : Nama Mahasiswa\tabularnewline NPM& : 555xxxx
\tabularnewline Jurusan& : Nama Jurusan\tabularnewline
Pembimbing& : Nama Pembimbing
\end{tabular}
\end{center}
\vspace{1.25cm}
\begin{center}
\bfseries PENULISAN ILMIAH \\
Diajukan Untuk Melengkapi Syarat \\
Mencapai Jenjang DIII/ Setara Sarjana Muda\\

\vspace{2.0cm}
%\vfill

Universitas Gunadarma\\
Tahun %DIGANTI TAHUN PENULISAN PI

\end{center}
%\vfill

\end{document}


%halaman lembar pengesahan, abstraksi, kata pengantar
\pagestyle{plain}
\pagenumbering{roman}
\setcounter{page}{2} %nilai halaman utk awal dari abstrak s/d ucapan terimakasih dlm romawi
\newpage
\addcontentsline{toc}{chapter}{LEMBAR PENGESAHAN}
\begin{center}
{\large \bf \centering LEMBAR PENGESAHAN}
\end{center}

\vspace{0.75cm}

\begin{tabbing}*
  \hspace{3.5cm}\=\hspace{0.25cm}  \=\hspace{0.5cm} \kill
  % \> for next tab, \\ for new line...
  Judul PI \> : \> DIGANTI DENGAN JUDUL PI \\
  Nama \> : \> DIGANTI DENGAN NAMA PENULIS PI \\
  NPM \> : \> DIGANTI DENGAN NPM PENULIS PI \\
  NIRM \> : \> DIGANTI DENGAN NIRM PENULIS PI \\
  Tanggal Sidang \> : \> DIGANTI DENGAN TANGGAL SIDANG PI \\
  Tanggal Lulus \> : \> DIGANTI DENGAN TANGGAL LULUS PI
\end{tabbing}


\begin{center}
\vspace{1cm}

{\bf Menyetujui}

\vspace{0.5cm}

{Pembimbing~ \hspace{5.0cm} ~Koordinator PI}%

\vspace{2.0cm}

{(NAMA PEMBIMBING)~ \hfill ~(NAMA KOORDINATOR PI)}%

\vspace{0.5cm}

 Ketua Jurusan\\
 \vspace{2.0cm}

(NAMA KETUA JURUSAN)

\end{center}
% menggabungkan file lembar pengesahan

%\documentclass{article}
%\begin{document}
\newpage %Abstract
\addcontentsline{toc}{chapter}{ABSTRAKSI}
\begin{center}
\begin{large}\textbf{ABSTRAKSI}\end{large}
\end{center}
\vspace{5mm} \textbf{Aldino Septa Nugraha}, 50402086\\
\textbf{"Pembuatan Web Pengisian KRS \emph{Online} Menggunakan PHP dan MySQL"}
\\
PI Fakultas Teknologi Industri, 2006 \\
Kata Kunci : KRS, OnLine, PHP, MySQL\\
(xi + 81 + Lampiran + Indeks) \\

Diketik sesuai dengan kebutuhan \\
Daftar Pustaka (1994-2005)  %%Diganti dengan tahun literatur yang digunakan
%\end{document}% menggabungkan file abstraksi

%menggabungkan file CV atau riwayat hidup ringkas
%
\newpage %CV
\addcontentsline{toc}{section}{Riwayat Hidup}

\begin{center}
\begin{large}
\textbf{Riwayat Hidup}\\
\end{large}
\end{center}
\vspace{5mm}

\begin{tabular}{ll}
Nama&
DIGANTI NAMA PENULIS\tabularnewline
Tanggal Lahir&
DIGANTI TEMPAT, TANGGAL LAHIR\tabularnewline
Pendidikan Formil&
DIGANTI PENDIDIKAN FORMIL TERAKHIR\tabularnewline
Pendidikan Non-Formil&
DIISI JIKA ADA DENGAN PENDIDIKAN NON-FORMIL TERAKHIR\tabularnewline
\end{tabular}


\newpage %Acknowledgment
\addcontentsline{toc}{chapter}{KATA PENGANTAR}
\begin{center}
\begin{large}\textbf{KATA PENGANTAR}\\\end{large}
\end{center}
\vspace{5mm}
Segala puji dan syukur penulis naikkan ke hadirat Tuhan yang Maha Kuasa yang telah memberikan berkat, anugerah dan karunia yang melimpah, sehingga penulis dapat menyelesaikan Penulisan Ilmiah ini pada waktu yang telah ditentukan.

Penulisan ilmiah ini disusun guna melengkapi sebagian syarat untuk memperoleh gelar setara Sarjana Muda Teknik Informatika Universitas Gunadarma. Adapun judul Penulisan Ilmiah ini adalah "Pembuatan Web Pengisian KRS \emph{Online}\\ Menggunakan PHP dan MySQL".

Walaupun banyak kesulitan yang penulis harus hadapi ketika menyusun penulisan ilmiah ini, namun berkat bantuan dan dorongan dari berbagai pihak, akhirnya tugas akhir ini dapat diselesaikan dengan baik. Untuk itu penulis tidak lupa mengucapkan terima kasih kepada:


Akhir kata, hanya kepada Tuhan jualah segalanya dikembalikan dan penulis sadari bahwa penulisan ini masih jauh dari sempurna, disebabkan karena berbagai keterbatasan yang penulis miliki. Untuk itu penulis mengharapkan kritik dan saran yang bersifat membangun untuk menjadi perbaikan di masa yang akan datang.

\vspace{0.5 cm}
\begin{flushright}
Depok, Juli 2006

\vspace{1.5 cm}
Penulis
\end{flushright}% mengabungkan file kata pengantar

%# akhir bagian administrasi # ==========================================

% #mulai membuat daftar isi, daftar gambar dan  daftar tabel # ==========
%\setcounter{page}{8} %set nilai halaman sesuai urutannya

% membuat daftar isi
\addcontentsline{toc}{chapter}{DAFTAR ISI}
\tableofcontents{}

%membuat daftar gambar
\listoffigures
\addcontentsline{toc}{chapter}{DAFTAR GAMBAR}%masih masalah

%% membuat daftar tabel
\listoftables
\addcontentsline{toc}{chapter}{DAFTAR TABEL}

\newpage %lampiran
\addcontentsline{toc}{chapter}{DAFTAR LAMPIRAN}
\begin{center}
\begin{large}\textbf{DAFTAR LAMPIRAN}\\\end{large}
\end{center}
\vspace{5mm} Lampiran 1 Listing Program
...............................................................................
10



% #akhir membuat daftar isi, daftar gambar dan  daftar tabel # ==========

% # mulai bagian isi # ==================================================

\newpage
\pagestyle{headings}
\pagenumbering{arabic} % jenis huruf arabic
\setcounter{page}{1} %mulai dari halaman 1


% \input{bab1/bab1.tex} %Bab 1

% \input{bab2/bab2.tex} %Bab 2

% \input{bab3/bab3.tex} %Bab 3

% \input{bab4/bab4.tex} %Bab 4




% diatas menunjukkan SubDir/nama file tex, bisa ditambah/dikurang sesuai kebutuhan

% # akhir bagian isi # ==================================================

% # mulai bagian Daftar Pustaka # ============================================
\newpage
\addcontentsline{toc}{chapter}{DAFTAR PUSTAKA} %memasukkan daftar pustaka di daftar isi
\bibliographystyle{IEEEtranS}
\bibliography{MainTemplatePI} %file menyimpan bibtex

% # akhir bagian referensi # ============================================

\newpage
\pagestyle{plain}
\newpage %lampiran
\addcontentsline{toc}{chapter}{LAMPIRAN}
\singlespacing
\begin{center}
\begin{large}\textbf{LAMPIRAN}\\\end{large}
\end{center}
\vspace{5mm}
Bisa diketik sesuai kebutuhan

\end{document}
%% Finish----------------
